\section{Introduction}

\subsection{Motivation}
Algorithmic analysis of interaction between people from sensor data
is a already a field that spans multiple disciplines. More recently,
with electronic devices in the mix, the problem is much more
multi-facted, as well as intriguing. Among humans, face-to-face
conversations are a very important form of social interaction. It is
one of the most natural and effective ways to fulfill our social
communication needs.

Computational analysis and prediction of group-device interactions are
tremendously valuable. This field of study promises to help us develop
solutions that address the analysis, understanding and interpretation
of human social behavior. Broadly, the goal of this project is to
build an ecosystem that has dynamically interacting devices and
actors. The components of the system which consists of humans and
devices enter and exit the system in a dynamic manner. The system
seemlessly keeps track of the participants and plugs them in and out
as they move in and out of the system.

Significant strides have been made in the field of computer vision
research recently. Many of them pertain to detecting, analyzing and
tracking human social interactions. This is partly due to the ready
availability of relatively good quality sensors and recent advances in
machine learning technologies. As such, we are tying to leverage such
research to advance our understanding of the role played by technology
in the field of \emph{human-computer interaction}. Particularly we aim
to detect, analyze, track and predict activities in groups and the
nature of their interaction. Of particular interest to is the
involvement of devices and how the human actors in a given scene
interact with the devides and other participants.

\subsection{Cross-device Interaction}
In their book \emph{Designing Connected Products: UX for the Consumer
  Internet of Things}, the authors state the following \emph{``in
  systems where functionality and interactons are distributed across
  more than one device, it's not enough to design individual UIs in
  isolation''} and that designers need to create coherent user
experiences across all the devices. This, according to the authors is
to develop a coherent understanding of the entire echosystem and how
the user may move between using different devices.

Our project aims to build such an ecosystem which involves human
actors, their devices and the intricate relationship of each of these
components with system.


